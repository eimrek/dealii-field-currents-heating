\documentclass[12pt,a4paper]{article}

\usepackage[utf8]{inputenc} % utf-8 encoding
%\usepackage{newtxtext,newtxmath} % Times New Roman
\usepackage{lmodern} % Latin modern font
\usepackage{siunitx} % for scientific notation AND units!
\usepackage{microtype} % better justification; always use with pdflatex!
\usepackage{booktabs} % midrule etc
\usepackage[margin=1.0in]{geometry} % nice margins
\usepackage{graphicx} % Images
\usepackage[T1]{fontenc} % font encoding
\usepackage{color}
\usepackage{hyperref} % clickable links
\usepackage{float} % for fixed floats (H)
\usepackage{subcaption} % figures side-by-side
\usepackage{amsmath} % better math structure
\usepackage{bm} % bold math
\usepackage{physics} % bra, ket and other symbols
\usepackage{setspace} % 1.5 line spacing
\onehalfspacing

% For documents in estonian (changes date; figure captions etc...)
%\usepackage[estonian]{babel} 

%\usepackage[backend=biber,sorting=none]{biblatex}
%\addbibresource{./ref.bib}

\numberwithin{figure}{section}
\numberwithin{table}{section}
\numberwithin{equation}{section}

\allowdisplaybreaks[1] % page breaks allowed in e.g. long align environ.

\usepackage{tikz} % flowcharts, other tikz figures
\usetikzlibrary{shapes,arrows,fit,calc}
\usepackage{pgfplots}
%\usepgfplotslibrary{clickable}
\pgfplotsset{
	compat=1.9,
	width=0.55\textwidth,
	tick style={line width=0.8pt},
	every axis plot/.append style={line width=0.8pt},
	every axis/.append style={line width=0.8pt}
	}

\newcommand{\note}[1]{\textcolor{red}{#1}} % notes, comments in red
\newcommand{\longnote}[1]{{\leavevmode\color{blue!40!black}#1}} % colored environm.

\begin{document}

\begin{center}
	{\LARGE Heat equation weak formulation} \\
\end{center}

% Use \section*{} for no number
\section{Time dependent heat equation}

\begin{align}
&\rho c_p \frac{\partial T}{\partial t} = \nabla \cdot \left( \kappa(T) \nabla T \right) + \sigma(T) (\nabla \varphi)^2 \\
&\vec{n} \cdot \left( \kappa(T) \nabla T \right) = Q(T) \text{ on $\Omega$}
\end{align}



\section{Euler implicit}

Assuming that conductivities have only previous time step values (otherwise the problem would be non-linear) and $\gamma = \rho c_p / \Delta t$:
\begin{equation}
\gamma (T_{n+1} - T_n) = \nabla \cdot \left( \kappa(T_n) \nabla T_{n+1} \right) + \sigma(T_n) (\nabla \varphi_{n+1})^2 
\end{equation}

Weak formulation (multiply by test function $\phi_i$ and integrate over the domain)
$$\gamma \int \phi_i (T_{n+1} - T_n) = \int \phi_i \nabla \cdot \left( \kappa(T_n) \nabla T_{n+1} \right) + \int \phi_i \sigma(T_n) (\nabla \varphi_{n+1})^2 $$
$$\gamma \int \phi_i (T_{n+1} - T_n) = \oint \phi_i \vec{n} \cdot \left( \kappa(T_n) \nabla T_{n+1} \right) - \int \nabla \phi_i \cdot \left( \kappa(T_n) \nabla T_{n+1} \right) + \int \phi_i \sigma(T_n) (\nabla \varphi_{n+1})^2 $$
$$\gamma \int \phi_i T_{n+1} + \int \nabla \phi_i \cdot \left( \kappa(T_n) \nabla T_{n+1} \right) = \gamma \int \phi_i T_n + \oint \phi_i Q(T_{n+1})  + \int \phi_i \sigma(T_n) (\nabla \varphi_{n+1})^2 $$

The heat boundary condition $Q(T_{n+1})$ should be set to $Q(T_{n})$, otherwise the problem still remains non-linear.

\section{Crank Nicolson}


Assuming that conductivities have only previous time step values (otherwise the problem would be non-linear) and $\gamma = \rho c_p / \Delta t$:
\begin{multline}
\gamma (T_{n+1} - T_n) = \frac{1}{2} ( \nabla \cdot \left( \kappa(T_n) \nabla T_{n+1} \right) + \sigma(T_n) (\nabla \varphi_{n+1})^2 + \nabla \cdot \left( \kappa(T_n) \nabla T_{n} \right) + \ldots \\ \ldots + \sigma(T_n) (\nabla \varphi_{n})^2 )
\end{multline}
\begin{equation}
2 \gamma (T_{n+1} - T_n) = \nabla \cdot \left( \kappa(T_n) \nabla T_{n+1} + \kappa(T_n) \nabla T_{n} \right) + \sigma(T_n) ( (\nabla \varphi_{n+1})^2 + (\nabla \varphi_{n})^2 ) 
\end{equation}

Weak formulation (multiply by test function $\phi_i$ and integrate over the domain)
\begin{multline}
2 \gamma \int \phi_i (T_{n+1} - T_n) = \int \phi_i \nabla \cdot \left( \kappa(T_n) \nabla T_{n+1} + \kappa(T_n) \nabla T_{n} \right) + \ldots \\ \ldots + \int \phi_i \sigma(T_n) ( (\nabla \varphi_{n+1})^2 (\nabla \varphi_{n})^2 )
\end{multline}
\begin{multline}
2 \gamma \int \phi_i (T_{n+1} - T_n) = \oint \phi_i \vec{n} \cdot \left( \kappa(T_n) \nabla T_{n+1} + \kappa(T_n) \nabla T_{n} \right) - \ldots \\ \ldots - \int \nabla \phi_i \cdot \left( \kappa(T_n) \nabla T_{n+1} + \kappa(T_n) \nabla T_{n} \right) + \int \phi_i \sigma(T_n) ( (\nabla \varphi_{n+1})^2 + (\nabla \varphi_{n})^2 )
\end{multline}
\begin{multline}
2 \gamma \int \phi_i T_{n+1} + \int \nabla \phi_i \cdot \left( \kappa(T_n) \nabla T_{n+1} \right) = 2 \gamma \int \phi_i T_n + \oint \phi_i (Q(T_n)+Q(T_{n+1})) - \ldots \\ \ldots - \int \nabla \phi_i \cdot \left( \kappa(T_n) \nabla T_{n} \right) + \int \phi_i \sigma(T_n) ( (\nabla \varphi_{n+1})^2 + (\nabla \varphi_{n})^2 )
\end{multline}

The heat boundary condition $Q(T_{n+1})$ should be set to $Q(T_{n})$, otherwise the problem still remains non-linear.



%\addcontentsline{toc}{section}{References}
%\pagebreak
%\printbibliography

\end{document}
